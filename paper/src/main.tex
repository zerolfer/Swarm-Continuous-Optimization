
% LLNCS macro package for Springer Computer Science

\documentclass[runningheads]{llncs}
%
\usepackage[bookmarks,hidelinks]{hyperref}
%
\usepackage{graphicx}
%
\usepackage{amsmath}
\usepackage[T1]{fontenc}
\usepackage[utf8]{inputenc}

\graphicspath{{img/}}

\newcommand{\refcruzada}[2]{\hyperref[#2]{#1~\ref{#2}}}
\newcommand{\comment}[1]{}

% Used for displaying a sample figure. If possible, figure files should
% be included in EPS format.
%
% If you use the hyperref package, please uncomment the following line
% to display URLs in blue roman font according to Springer's eBook style:
% \renewcommand\UrlFont{\color{blue}\rmfamily}
%
%
\begin{document}
    %
    \title{Contribution Title\thanks{Trabajo desarrollado como proyecto final de la asignatura \textit{Agentes inteligentes y sistemas multiagente} del Máster Universitario en Inteligencia Artificial de la Universidad Politécnica de Madrid}}
    %
    %\titlerunning{Abbreviated paper title}
    % If the paper title is too long for the running head, you can set
    % an abbreviated paper title here
    %
    \author{Sergio Flórez Vallina}
    %

    % First names are abbreviated in the running head.
    % If there are more than two authors, 'et al.' is used.
    %
    \institute{Universidad Politécnica de Madrid, España\\
    \email{s.florez@alumnos.upm.es}\\
    }
    %
    \maketitle              % typeset the header of the contribution
    %
    \begin{abstract}
        The abstract should briefly summarize the contents of the paper in
        150--250 words.

        \keywords{First keyword  \and Second keyword \and Another keyword.}
    \end{abstract}
    %
    %
    %

    \section{Introducción}
    En los últimos años se ha observado que existen diversos problemas cuya naturaleza requiere de un enfoque
    multi-agente, pues un único agente no es capaz de resolverlos.
    Utilizando la Inteligencia Colectiva (CI), ya mediante una comunicación directa o indirecta entre los agentes,
    se han podido resolver problemas tanto en el ámbito de la robótica, tales como la planificación de rutas o
    construcción de estructuras complejas~\cite{collectiveConstruction}, así como en el ámbito de la optimización,
    esto es, de encontrar la mejor solución --o al menos, una de buena calidad-- para un problema dado.
    En este artículo, analizaremos un posible sistema multiagente propuesto inicialmente en~\cite{initialPaper}
    con el objetivo de hallar la posición óptima de un espacio de búsqueda y su comportamiento frente a obstáculos físicos.
    La descripción del problema y las técnicas empleadas para su solución se encuentra en el
    \refcruzada{Apartado}{sec:descripcion}, y los resultados experimentales y el ajuste paramétrico en
    el \refcruzada{Apartado}{sec:}. %TODO poner referencias!!

    \section{Descripcion del problema}
    \label{sec:descripcion}

    Tal y como se ha introducido anteriormente, el problema a resolver se basa en realizar una búsqueda dentro de un
    espacio bidimensional limitado a un cierto tamaño. Trataremos de maximizar una función conocida $f(x,y)$, sujeta
    a las restricciones de espacio impuestas inicialmente, es decir, al tamaño del mapa por el que se moverán los agentes.
    Para su desarrollo, se ha utilizado como base la técnica propuesta por~\cite{initialPaper}, sin embargo ésta no
    contempla el uso de obstáculos --ciertas posiciones del espacio que no pueden ser traspasados por el robot-- dentro
    del espacio de búsqueda, por lo que se ha extendido y adaptado.
    En el \refcruzada{Apartado}{3} se muestran algunos ejemplos que lo ilustran, utilizando tanto obstáculos como
    diferentes funciones objetivo %TODO referencias


    \subsection{El espacio de búsqueda}
    La representación propuesta en~\cite{initialPaper} consiste en discretizar el espacio de búsqueda en forma de una
    cuadrícula (\textit{grid}). Ésta es una de las formas clásicas más sencillas de representar un espacio de búsqueda,
    sin embargo, tal y como se apuntan en~\cite{AIRobotics}, este tipo de representación conlleva a una
    \textit{digitalización} o discretización del espacio con una cierta granularidad dada. A mayor granulidad,
    mayor fidelidad en la representación del espacio. En el artículo original, se utiliza una única granularidad
    sobre una cuadrícula $10 \times 10$. En el \refcruzada{Apartado}{3} se prueban diferentes tamaños. %TODO es esto cierto??

    Podemos considerar el espacio de búsqueda a efectos prácticos, como una colina tridimensional donde la tercera
    dimensión representa la intensidad de una señal que sigue una función (no necesariamente lineal). La finalidad
    del sistema, es que los agentes colaboren para alcanzar el punto más álgido.

    \subsection{La estrategia de resolución}

    Los agentes llevan a cabo movimientos sencillos empleando información en común entre sí, pero de forma indirecta,
    a través del medio. Éste tipo de soluciones reciben el nombre de sistemas basados en
    \textit{estigmergia}~\cite{stigmergy}.
    % lo muevo como parrafo para evitar un bad-box
    Como buen sistema multiagente, podemos observar una capacidad de auto-organización (\textit{self-organization}),
    mediante el cual emerge un comportamiento de características más complejas, que en este caso consisten en localizar
    y converger a un lugar concreto del mapa.

    En este enfoque, los agentes se moverán libremente por el espacio, sin limitarse a la cuadrícula, y depositarán
    en las respectivas celdas sobre las que se encuentre, una información dada que llamaremos feromonas, que estarán
    representadas mediante un vector cuya dirección y sentido sea la más prometedora del espacio y cuyo módulo será
    proporcional a la mejora que aporta dicho movimiento al valor de la función objetivo.

    El depósito de dichas feromonas se lleva a cabo siguiente la siguiente estrategia, que está basada en el
    comportamiento de un sistema PSO (\textit{Particle Swarm Optimization} \cite{PSO}: cada agente se moverá a una
    cierta velocidad, que se verá afectada por la información del entorno según la siguiente fórmula:


    \[
        \vec{V_i^{k+1}} = \omega \cdot \vec{V_i^k} + c_1 \cdot r_1(\vec{P_i} - \vec{V_i^k}) + c_2 \cdot r_2 \cdot \vec{s_n}
    \]

    Donde $\omega$, $c_1$ y $c_2$ son parámetros del algoritmo que denominaremos el factor de inercia, el factor de
    aprendizaje local y el factor de aprendizaje social respectivamente. Y $r_1, r_2 \sim U[0,1]$ son dos factores
    aleatorios independientes uniformemente distribuidos.
    El agente almacenará locamente la mejor solución hallada por el mismo hasta el momento, $\vec{P_i}$.
    Y por último, $\vec{s_n}$ es el vector de feromonas obtenido de la celda del mapa correspondiente a la
    posición $X_A$ del robot.

    \begin{figure}
        \includegraphics[scale=0.5]{"vectores"}
        \centering
        \caption{}
        \label{fig:1}
    \end{figure}

    Si observamos la \refcruzada{Fig.}{fig:1}, podemos ver cuales serían los puntos A y B durante un posible movimiento
    de un agente. $\overline{AC}$ sería un posible movimiento de otro agente que ha pasado previamente por la misma
    celda (que contiene al punto $A$) y ha escrito en ella el vector $s_2$

    El vector resultante $s$ se calcula en base a la información heurística dada por la intensidad de la señal de la
    función objetivo según la formula
    \begin{equation}
        \label{ec:1}
        \includegraphics[scale=0.5]{"formula feromonas"}
        \centering
    \end{equation}
    Nótese que la dirección vendrá dada posición hacia la que se está moviendo el robot, mientras que el sentido y
    el módulo del mismo dependerá de la función objetivo de forma directamente proporcional a ésta.

    Una vez calculado el nuevo rastro de feromonas a depositar, se utilizará la siguiente formula para combinar las
    feromonas ya existentes en la celda leídas anteriormente con la nueva generada, basada en este caso, en la
    evaporación de las feromonas en ACO (\textit{Ant Colony Optimization})~\cite{ACO}:
    \[
        \vec{s_n^{m+1}=(1-c_d)\cdot\vec{s_n^m+c_a\cdot\vec{s_i}}}
    \]
    En este caso, el agente $i$ es el 1, pues es $s_1$ el vector calculado. $c_a$ es el factor de superposición
    y $c_d$ de evaporación, nuevamente parámetros del algoritmo.

    La nueva posición del agente se calculará como la suma vectorial de su posición con la nueva velocidad,
    \[  \vec{ X_i^{k+1} } = \vec{X_i^k} + \vec{ V_i^{k+1} }  \]

    Por último, para poder evitar los obstáculos, en caso de que el agente no sea capaz de mejorar su mejor solución
    local tras un numero máximo de iteraciones, se generará una nueva velocidad con dirección y sentido arbitrario,
    y con módulo $V_{max}$, la velocidad máxima posible de los agentes. En caso de no poder sortear el obstáculo, se
    trata durante otro cierto numero de iteraciones, de rotar el vector velocidad generado anteriormente una cantidad
    de grados, en este caso de $10$ (véase~\cite{referencedPaper}).


    \section{Resultados experimentales}

    % TODO: poner tablas con los valores de todos los parámetros del sistema


    \subsection{Variaciones en la función objetivo}


    \subsection{Pruebas con obstáculos}













    \comment{
    \section{First Section}
    \subsection{A Subsection Sample}
    Please note that the first paragraph of a section or subsection is
    not indented. The first paragraph that follows a table, figure,
    equation etc. does not need an indent, either.

    Subsequent paragraphs, however, are indented.

    \subsubsection{Sample Heading (Third Level)} Only two levels of
    headings should be numbered. Lower level headings remain unnumbered;
    they are formatted as run-in headings.

    \paragraph{Sample Heading (Fourth Level)}
    The contribution should contain no more than four levels of
    headings. Table~\ref{tab1} gives a summary of all heading levels.

    \begin{table}
        \caption{Table captions should be placed above the
        tables.}\label{tab1}
        \begin{tabular}{|l|l|l|}
            \hline
            Heading level &  Example & Font size and style\\
            \hline
            Title (centered)  & {\Large\bfseries Lecture Notes} 				 & 14 point, bold   \\
            1st-level heading & {\large\bfseries 1 Introduction} 				 & 12 point, bold   \\
            2nd-level heading & {\bfseries 2.1 Printing Area} 					 & 10 point, bold   \\
            3rd-level heading & {\bfseries Run-in Heading in Bold.} Text follows & 10 point, bold   \\
            4th-level heading & {\itshape Lowest Level Heading.} Text follows 	 & 10 point, italic \\
            \hline
        \end{tabular}
    \end{table}


    \noindent Displayed equations are centered and set on a separate
    line.
    \begin{equation}
        x + y = z
    \end{equation}
    Please try to avoid rasterized images for line-art diagrams and
    schemas. Whenever possible, use vector graphics instead (see
    Fig.~\ref{fig1}).

    \begin{figure}
        \includegraphics[width=\textwidth]{fig1.eps}
        \caption{A figure caption is always placed below the illustration.
        Please note that short captions are centered, while long ones are
        justified by the macro package automatically.} \label{fig1}
    \end{figure}

    \begin{theorem}
        This is a sample theorem. The run-in heading is set in bold, while
        the following text appears in italics. Definitions, lemmas,
        propositions, and corollaries are styled the same way.
    \end{theorem}
    %
    % the environments 'definition', 'lemma', 'proposition', 'corollary',
    % 'remark', and 'example' are defined in the LLNCS documentclass as well.
    %
    \begin{proof}
        Proofs, examples, and remarks have the initial word in italics,
        while the following text appears in normal font.
    \end{proof}
    For citations of references, we prefer the use of square brackets
    and consecutive numbers. Citations using labels or the author/year
    convention are also acceptable. The following bibliography provides
    a sample reference list with entries for journal
    articles~\cite{ref_article1}, an LNCS chapter~\cite{ref_lncs1}, a
    book~\cite{ref_book1}, proceedings without editors~\cite{ref_proc1},
    and a homepage~\cite{ref_url1}. Multiple citations are grouped
    \cite{ref_article1,ref_lncs1,ref_book1},
    \cite{ref_article1,ref_book1,ref_proc1,ref_url1}.
    }
    %
    % ---- Bibliography ----
    %
    % BibTeX users should specify bibliography style 'splncs04'.
    % References will then be sorted and formatted in the correct style.
    %
    \bibliographystyle{splncs04}
    \bibliography{main}

\end{document}
